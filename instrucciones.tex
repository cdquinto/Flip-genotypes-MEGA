\documentclass[12pt]{report}
\usepackage{geometry}
\geometry{letterpaper, margin=0.7in}
\usepackage{tgtermes}
\usepackage{graphicx}
\usepackage{verbatim}
\usepackage[spanish]{babel}
\usepackage{subcaption}
\usepackage{subfig} 
\usepackage{multirow}
\usepackage{hyperref}
\usepackage[usenames,dvipsnames,svgnames,table]{xcolor}
\usepackage{ tipa }
\usepackage{ textcomp }
\usepackage{ amssymb }
\usepackage{ragged2e}
%%%%%%%%%%%%%%%%%%%%

\begin{document}

\today

Consuelo Dayzu Quinto

\begin{center}
\large \textbf{Procesamiento de los datos del microarreglo MEGA}
\end{center}
\\

\begin{enumerate}
\item Filtrar los genotipos por call frequency (0.95) y remover sitios que no son SNPs (ej. indels).
\item Generar los archivos plink desde Genome Studio.
\item Migrar estos archivos al mismo directorio que contiene los siguientes archivos:
\begin{itemize}
\item \colorbox{Lavender}{list\_snps\_to\_change\_alleles.txt}
\item \colorbox{Lavender}{new\_positions\_snps.txt}
\item \colorbox{Lavender}{MEGA\_Consortium\_v2\_15070954\_A1\_b138\_rsids.txt}
\item \colorbox{Lavender}{1KG\_Omni.bed; 1KG\_Omni.fam; 1KG\_Omni.bim}
\item \colorbox{Lavender}{convert\_MEGA\_alleles\_sentli.pl}: este script remueve SNPs mapeados en el cromosoma 0, SNPs duplicados (por posicion fisica y por ID), cambia los alelos, actualiza la posicion fisica de dos SNPs, renombra los SNPs, y finalmente calcula la concordancia de genotipos de alguno de estos controles: HG01946, HG01935, NA21440, NA21317.\\ IMPORTANTE 1: Los controles se tienen que llamar de esta maner o habra un error en el programa. \\ IMPORTANTE 2: La concordancia debe de ser mayor a 0.95. 
\end{itemize}
\item Correr el script: \colorbox{Lavender}{perl convert\_MEGA\_alleles\_sentli.pl archivo.ped}
\item Este programa genera los siguientes archivos:
\begin{itemize}
\item \colorbox{Lavender}{archivo.flip.ped} y \colorbox{Lavender}{archivo.flip.map}: archivos PLINK finales.
\item \colorbox{Lavender}{duplicate\_snps\_archivo.txt}: ID de los SNPs que tienen el mismo nombre.
\item \colorbox{Lavender}{duplicate\_positions\_archivo.txt}: ID de los SNPs que tienen la misma posicion.
\end{itemize}
\item Aparte de los archivos procesados, hay dos archivos que se pueden dar a los usuarios:
\begin{itemize}
\item \colorbox{Lavender}{multiallelic\_SNPs.txt} tiene la lista de los SNPs que no son bialelicos (9,583 SNPs). 
\item \colorbox{Lavender}{complementary\_snps.txt} tiene la lista de SNPs cuyos alelos son C/G o A/T (2,036 SNPs).
\end{itemize}
\end{enumerate}
\end{document}
