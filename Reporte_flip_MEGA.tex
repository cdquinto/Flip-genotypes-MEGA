\documentclass[10pt]{report}
\usepackage{geometry}
\geometry{letterpaper, margin=0.7in}
\usepackage{tgtermes}
\usepackage{graphicx}
\usepackage{verbatim}
\usepackage[spanish]{babel}
\usepackage{subcaption}
\usepackage{subfig} 
\usepackage{multirow}
\usepackage{hyperref}
\usepackage[usenames,dvipsnames,svgnames,table]{xcolor}
\usepackage{ tipa }
\usepackage{ textcomp }
\usepackage{ amssymb }
\usepackage{ragged2e}
%%%%%%%%%%%%%%%%%%%%

\begin{document}

\today

Consuelo Dayzu Quinto

\begin{center}
\large \textbf{Cambio de alelos de acuerdo a la cadena FORWARD y REVERSE en los SNPs del microarreglo MEGA}
\end{center}
\\

Usando el manifiesto Multi-EthnicGlobal\_A1.bpm  y los datos crudos de la primera corrida de Maria Avila (maavp1v1), se obtuvo un reporte final (maavp1v1.raw\_FinalReport.txt) que contiene la informacion de los alelos en la cadena Forward para cada SNP en el microarreglo MEGA. \\

Con la informacion de este reporte mas informacion de la base de datos dbSNP y 1KG, hice una lista\\
 \colorbox{Lavender}{list\_snps\_to\_change\_alleles.txt}. Esta lista contiene el identificador del SNP, los alelos detectados por GenomeStudio, los alelos de dbSNP y 1KG, y la posicion fisica para 664,465 SNPs. \\

Archivo list\_snps\_to\_change\_alleles.txt:\\
\colorbox{Lavender}{JHU\_8.51371909	A	 G	 T	 C	 51371910}\\
\colorbox{Lavender}{JHU\_9.117096517	A	 G	 T	 C	 0}\\
\colorbox{Lavender}{rs55873141	C	 G	 G	 C	 113845176}	\\
\colorbox{Lavender}{JHU\_2.83159627	A	 G	 T	 C	 83159628}	\\
\colorbox{Lavender}{rs10806671	A	 G	 T	 C	 170270028}	\\
\colorbox{Lavender}{2:630995-T-G	A	 C	 T	 G	 630995}	\\
\colorbox{Lavender}{rs6079035	A	 C	 T	 G	 13368741}	\\
\colorbox{Lavender}{rs2088629	A	 G	 T	 C	 133287048}	\\
\colorbox{Lavender}{rs2066705	A	 G	 T	 C	 25937004}	\\

Escribi un script en perl \colorbox{Lavender}{convert\_MEGA\_alleles.pl} para procesar los datos crudos provenientes de GenomeStudio que realiza las siguientes cosas:
\begin{itemize}
 \item Remueve SNPs mapeados en el cromosoma cero ($\sim$ 16,749 sin cluster; $\sim$ 10,791 con cluster).
 \item Remueve SNPs duplicados por ID y por posicion fisica. 
 \item Cambia los alelos de acuerdo a la lista \colorbox{Lavender}{list\_snps\_to\_change\_alleles.txt} (664,465 SNPs).
 \item Actualiza la posicion fisica de dos SNPs: rs9522257 y rs9480186 \colorbox{Lavender}{new\_positions\_snps.txt}
 \item Renombra los SNPs con la lista \colorbox{Lavender}{MEGA\_Consortium\_v2\_15070954\_A1\_b138\_rsids.txt}
 \end{itemize}

Para correr este script solo se necesita un archivo PED:\\
\colorbox{Lavender}{perl convert\_MEGA\_alleles.pl archivo.ped}\\

Este script genera:
\begin{itemize}
\item archivo.flip.ped y archivo.flip.map
\item duplicate\_snps\_archivo.txt: ID de los SNPs duplicados
\item duplicate\_positions\_entrenamiento\_archivo.txt: ID de los SNPs que tienen la misma posicion
\item to\_remove.txt: ID de los SNPs duplicados que se excluyen del archivo map original
\end{itemize}
\\

NOTA IMPORTANTE: Los archivos 
\colorbox{Lavender}{list\_snps\_to\_change\_alleles.txt; new\_positions\_snps.txt;}\\
\colorbox{Lavender}{MEGA\_Consortium\_v2\_15070954\_A1\_b138\_rsids.txt} tienen que estar en el mismo directorio donde se encuentren los archivos de plink, y se requiere de la ultima version de plink (plink 1.9).

Aparte de los archivos post-procesados, hay dos archivos que se pueden dar a los usuarios:
\begin{itemize}
\item multiallelic\_SNPs.txt tiene la lista de los SNPs que no son bialelicos (9,583 SNPs). 
\item complementary\_snps.txt tiene la lista de SNPs cuyos alelos son C/G o A/T (2,036 SNPs).
\end{itemize}

\end{document}
